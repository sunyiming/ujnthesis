\usepackage{pdfpages, hyperref} % 页面间距
\usepackage[a4paper,left=3cm,right=2.5cm,top=2.5cm,bottom=2cm]{geometry}
%\usepackage{titlesec}

\usepackage{fontspec} % 英文字体
\setmainfont{Times New Roman}

\usepackage{fancyhdr} % 全局页眉和边缘线
\pagestyle{fancy}
\fancyhead{}
\renewcommand\headrulewidth{1.5pt}
\renewcommand\footrulewidth{1.5pt}
\chead{济南大学毕业设计}

\ctexset{
	section/aftername = {\enspace},
	subsection/aftername = {\enspace},
	subsubsection/aftername = {\enspace}
}

\usepackage{titletoc} % 目录格式设置
\titlecontents{section}
[1em]{\normalsize\vspace{0}}{\contentslabel{1em}}{\hspace*{-1em}}{~\titlerule*[0.4pc]{.}~\contentspage}
\titlecontents{subsection}
[4em]{\normalsize\vspace{0}}{\contentslabel{2em}}{\hspace*{0}}{~\titlerule*[0.4pc]{.}~\contentspage}
\titlecontents{subsubsection}
[6.5em]{\normalsize\vspace{0}}{\contentslabel{2.5em}}{\hspace*{0}}{~\titlerule*[0.4pc]{.}~\contentspage}

\renewcommand{\contentsname}{\textbf{\zihao{4}目\qquad 录}}


\fancypagestyle{ujnabstract} % 定义中文摘要格式
{
	\pagenumbering{Roman} % 罗马数字页码
	\fancyfoot[C]{- \thepage{} -} % 页码居中左右两杠
}
\fancypagestyle{ujnabstract*} % 定义英文摘要格式
{
	\pagenumbering{Roman} % 罗马数字页码
	\fancyfoot[C]{- \thepage{} -} % 页码居中左右两杠
}
\fancypagestyle{ujncontent} % 定义目录格式
{
	\fancyfoot{} % 清除页脚
	\ctexset{section/nameformat = \large\heiti\centering} % 标题格式
}
\fancypagestyle{ujnbody} % 定义正文格式
{
	\ctexset{section/titleformat = \large\heiti\centering} % 标题格式
	\pagenumbering{arabic} % 数字页码
	\fancyfoot[C]{- \thepage{} -} % 页码居中左右两杠
}
