\begin{ujnabstract}
\section[摘要]{摘\qquad 要}
% 以下为中文摘要内容,把点引线以下的文字替换成你的就好
% ·································································································
%近日,由教务处 教学质量与评估办公室编写的《“四路并举”赋能专业建设新样态》成功入选教育部《全国普通高校本科教育教学质量报告》(以下简称《质量报告》)应用典型案例,并将编入由高等教育出版社出版的《数智助鉴 以鉴促强——落实质量主体责任高校在行动》一书。
%
%《质量报告》是在教育部教育督导局指导下,由教育部教育质量评估中心会同有关高校、教育研究机构等单位组建的学术团队联合研制,系统反映我国普通高校本科教育教学质量发展状况的年度报告。应用案例征集活动旨在选树一批高校教学质量管理优秀典型,巩固本科教育教学改革成果,完善体制机制建设。
%
%入选案例《“四路并举”赋能专业建设新样态》系统总结了学校近年来深化专业综合改革的建设举措以及成效。以“需求驱动、质量带动、产业拉动、内外联动”为引擎,围绕“四新”专业建设主线,推动本科专业供给侧改革,优化专业布局动态调整,积极搭建产教融合育人平台,持续推进专业建设高水平发展,提高专业人才培养质量。
%
%近年来,学校高度重视本科教育教学工作,始终坚持以习近平新时代中国特色社会主义思想为指导,落实立德树人根本任务,推动学校本科教育内涵建设和高质量发展。学校将充分学习和研究《质量报告》,进一步扎实做好本科教育教学质量常态监测、编制发布本科教学质量报告和专业人才培养状况报告等工作。以数据分析报告为重要依据,积极引导各专业明确定位、强化特色、加强优势、提升质量,构筑高水平人才培养体系,不断提升育人成效。
近年来,学校高度重视本科教育教学工作,始终坚持以习近平新时代中国特色社会主义思想为指导,落实立德树人根本任务,推动学校本科教育内涵建设和高质量发展。学校将充分学习和研究《质量报告》,进一步扎实做好本科教育教学质量常态监测、编制发布本科教学质量报告和专业人才培养状况报告等工作。以数据分析报告为重要依据,积极引导各专业明确定位、强化特色、加强优势、提升质量,构筑高水平人才培养体系,不断提升育人成效。
近年来,学校高度重视本科教育教学工作,始终坚持以习近平新时代中国特色社会主义思想为指导,落实立德树人根本任务,推动学校本科教育内涵建设和高质量发展。学校将充分学习和研究《质量报告》,进一步扎实做好本科教育教学质量常态监测、编制发布本科教学质量报告和专业人才培养状况报告等工作。以数据分析报告为重要依据,积极引导各专业明确定位、强化特色、加强优势、提升质量
% ·································································································
% 以下为关键词的设置,把「关键词n」(n=1,2,3,...)替换成你的关键词就好
\small\cnkeywords{电力系统;关键词2;关键词3}
\end{ujnabstract}
