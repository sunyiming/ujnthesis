当今世界,以信息技术产业为代表的高新技术产业得到了迅猛发展,推动了全球产业结构转型和优化升级,
带来了人类生产生活方式的深刻变化. 进入 21 世纪,信息技术日新月异,
其普及应用对经济、政治、社会、文化、军事发展的影响更加深刻,
信息技术产业已经成为衡量一个国家或地区综合国力、国际竞争力和现代化程度的重要标志。
如何全面把握当代信息技术发展趋势,明确我国未来信息技术产业的发展思路和政策取向,
这是需要我们认真研究和思考的重大问题。

能源是人类生存和发展的重要物质基础,也是当今国际政治、经济、军事、外交关注的焦点。
中国经济社会持续高速发展,离不开有力的能源保障。
在经济全球化深入发展和中国现代化加快推进的大背景下,如何认识能源发展趋势,
选择什么样的能源发展战略,采取什么样的政策措施,是一个非常重要的问题,需要认真加以思考。
